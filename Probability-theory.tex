\documentclass[UTF8]{ctexart}

\usepackage{amsmath}

\begin{document}
	% 标题
	\title{概率论与数理统计总结}
	\author{刘阳}
	\date{\today}
	\maketitle
	
	% 目录
	\tableofcontents
	
	% 内容
	\section{第一章\ 概率论的基本概念}
	
	\subsection{随机试验}
	
	\begin{itemize}
		\item [随机试验:] {
			\begin{enumerate}
				\item [1.] {可以在相同条件下重复地进行;}
				\item [2.] {每次试验的可能结果不止一个,并且能事先明确实验的所有可能结果;}
				\item [3.] {进行一次实验之前不能确定哪一个结果会出现.}
			\end{enumerate}
		}
	\end{itemize}

	\subsection{样本空间、随机事件}
	
	\subsubsection{样本空间}
	
	{随机试验E的所有可能结果组成的集合称为 \textbf{随机试验}.}
	
	{样本空间的元素,即E的每个结果称为 \textbf{样本点}.}
	\subsubsection{随机事件}
	
	{试验E的样本空间S的子集称为 \textbf{随机事件},简称 \textbf{事件}.}
	
	{每次试验中,当且仅当这一子集的一个样本点出现称为 \textbf{事件发生}.}
	
	{有一个样本点组成的单点集称为 \textbf{基本事件}.}
	
	{样本空间S包含所有的样本点,它是S自身的子集,在每次试验中它总是发生的,S成为 \textbf{必然事件}.}
	
	{空集 $ \emptyset $ 不包含任何样本点,它也作为样本空间的子集,他在每次试验中都不发生, $ \emptyset $ 称为 \textbf{不可能事件}.}
	
	\subsubsection{事件间的关系与事件的运算}
	
	\begin{enumerate}
		\item [1.] {若 $ A \subset B $ ,则称事件 $ B $ 包含事件 $ A $ ,这指的是事件 $ A $ 发生必导致事件 $ B $ 发生. \\ 
					若 $ A \subset B $ 且 $ B \subset A $ ,即 $ A = B $ ,则称事件 $ A $ 与事件 $ B $  \textbf{相等}.}
		\item [2.] {事件 $ A \cup B=\{x|x \in A 或 x \in B\} $ 称为事件 $ A $ 与事件 $ B $ 的 \textbf{和事件}.当且仅当 $ A,B $ 中至少一个发生时,事件 $ A \cup B $ 发生. \\ 
					类似地,称 $ \bigcup\limits_{k=1}^{n}A_{k} $ 为 $ n $ 个事件 $ A_{1},A_{2}, \cdots ,A_{n} $ 的 \textbf{和事件};称 $ \bigcup\limits_{k=1}^{\infty}A_{k} $ 为可列个事件 $ A_{1},A_{2}, \cdots $ 的和事件.}
		\item [3.] {事件 $ A \cap B=\{x|x \in A 且 x \in B\} $ 称为事件 $ A $ 与事件 $ B $ 的 \textbf{积事件}.当且仅当 $ A,B $ 同时发生时,事件 $ A \cap B $ 发生. $ A \cap B $ 也记作 $ AB $ . \\ 
					类似地,称 $ \bigcap\limits_{k=1}^{n}A_{k} $ 为 $ n $ 个事件 $ A_{1},A_{2}, \cdots ,A_{n} $ 的 \textbf{积事件};称 $ \bigcap\limits_{k=1}^{\infty}A_{k} $ 为可列个事件 $ A_{1},A_{2}, \cdots $ 的积事件.}
		\item [4.] {事件 $ A-B=\{x|x \in A 且 x \notin B\} $ 称为事件 $ A $ 与事件 $ B $ 的 \textbf{差事件}.当且仅当 $ A $ 发生、 $ B $ 不发生时事件 $ A-B $ 发生.}
		\item [5.] {若 $ A \cap B= \emptyset $ 则称事件 $ A $ 与事件 $ B $ 是 \textbf{互不相容的},或 \textbf{互斥的}.这指的是事件 $ A $ 与事件 $ B $ 不能同时发生.基本事件也是两两互不相容的.}
		\item [6.] {若 $ A \cup B=S $ 且 $ A \cap B= \emptyset $ ,则称事件 $ A $ 与事件 $ B $ 互为 \textbf{逆事件}. 又称事件 $ A $ 与事件 $ B $ 互为 \textbf{对立事件}.这指的是对每次试验而言,事件 $ A,B $ 中必有一个发生,且仅有一个发生. $ A $ 的对立事件记为 $ \bar{A} $. $ \bar{A} =S-A $ }
	\end{enumerate}

	\begin{itemize}
		\item [交换律:] { $ A \cup B=B \cup A $ ; \\ $  A \cap B=B \cap A  $ .}
		\item [结合律:] { $ A \cup (B \cup C)=(A \cup B) \cup C $ ; \\ $  A \cap (B \cap C)=(A \cap B) \cap C $ .}
		\item [分配律:] { $ A \cup (B \cap C)=(A \cup B) \cap (A \cup C) $ ; \\ $ A \cap (B \cup C)=(A \cap B) \cup (A \cap C) $ .}
		\item [德摩根律:] { $ \bar{A \cup B} = \bar{A} \cap \bar{B} $ ; $ \bar{A \cap B} = \bar{A} \cup \bar{B} $ .}
	\end{itemize}
	
	\subsection{频率与概率}
	
	\subsubsection{频率}
	
	{在相同条件下,进行了 $ n $ 次试验,在这 $ n $ 次试验中,事件 $ A $ 发生的次数为 $ n_{A} $ 称为事件 $ A $ 发生的 \textbf{频数}.比值 $ \frac{n_{A}}{n} $ 称为事件 $ A $ 发生的 \textbf{频率},并记成 $ f_{n}(A) $ }
	
	\begin{itemize}
		\item [性质:] {
			\begin{enumerate}
				\item [1.] { $ 0 \le f_{n}(A) \le 1 $ ;}
				\item [2.] { $ f_{n}(S)=1 $ ;}
				\item [3.] { 若 $A_{1},A_{2}, \cdots ,A_{k} $ 是两两互不相容的事件,则 $ f_{n}(A_{1} \cup A_{2} \cup \cdots \cup A_{k})=f_{n}(A_{1})+f_{n}(A_{2})+ \cdots +f_{n}(A_{k}) $ .}
			\end{enumerate}
		}
	\end{itemize}

	\subsubsection{概率}
	
	{设 $ E $ 是随机试验, $ S $ 是它的样本空间.对于 $ E $ 的每一事件 $ A $ 赋予一个实数,记为 $ P(A) $ ,称为事件 $ A $ 的 \textbf{概率},如果集合函数 $ P( \cdot ) $ 满足以下条件:}
	
	\begin{enumerate}
		\item [1.] { \textbf{非负性:} 对于每一个事件 $ A $ ,有 $ P(A) \ge 0 $ ;}
		\item [2.] { \textbf{规范性:} 对于必然事件 $ S $ ,有 $ P(S)=1 $ ;}
		\item [3.] { \textbf{可列可加性:} 设 $ A_{1},A_{2}, \cdots $ 是两两互不相容的事件,即对于 $ A_{i}A_{j}= \emptyset $ , $ i \neq ,i,j=1,2, \cdots , $ 有 $ P(A_{1} \cup A_{2} \cup \cdots)=P(A_{1})+P(A_{2})+ \cdots $ .}
	\end{enumerate}

	\begin{itemize}
		\item [性质:] {
			\begin{enumerate}
				\item [1.] { $ P( \emptyset )=0 $ .}
				\item [2.] {若 $ A_{1},A_{2}, \cdots ,A_{n} $ 是两两互不相容的事件,则有 $ P(A_{1} \cup A_{2} \cup \cdots \cup A_{n})=P(A_{1})+P(A_{2})+ \cdots +P(A_{n}) $ .}
				\item [3.] {设 $ A,B $ 是两个事件,若 $A \subset B$ ,则有 $ P(B-A)=P(B)-P(A) $ ; $ P(B) \ge P(A) $ .}
				\item [4.] {对于任一事件 $ A $ , $ P(A) \le 1 $ .}
				\item [5.] {对于任意两事件 $ A,B $ 有 $ P(A \cup B)=P(A)+P(B)-P(AB) $ .}
			\end{enumerate}
		}
	\end{itemize}
	
	\subsection{等可能概型(古典概型)}
	
	\begin{enumerate}
		\item [1.] {试验的样本空间只包含有限个元素.}
		\item [2.] {试验中每个基本事件发生的可能性相同.}
	\end{enumerate}

	{具有以上两个特点的试验称为 \textbf{等可能概型}.它在概率论发展初期曾是主要的研究对象,所以也成为 \textbf{古典概型}.}
	
	{设试验的样本空间为 $ S=\{e_{1},e_{2}, \cdots ,e_{n}\}$ 若 $ A $ 包含 $ k $ 个基本事件则有: \\ $$ P(A)=\sum\limits_{j=1}^{k}P(\{e_{i_{j}}\})= \frac{k}{n} = \frac{A \mbox{包含的基本事件数} }{S \mbox{中基本事件的总数} } $$ }
	
	{ \textbf{超几何分布}的概率公式: $$ p= \frac{ \binom{D}{k} \binom{N-D}{n-k} }{ \binom{N}{n} } $$ }
	
	\subsection{条件概率}
	
	\subsubsection{条件概率}
	
	{设 $ A,B $ 是两个事件,且 $ P(A)>0 $ ,称 $ P(B|A)= \frac{P(AB)}{P(A)} $ 为在事件 $ A $ 发生下事件 $ B $ 发生的 \textbf{条件概率}.}
	
	\subsubsection{乘法定理}
	
	{设 $ P(A)>0 $ 则有 $ P(AB)=P(B|A)P(A) $ .}
	
	\subsubsection{全概率公式和贝叶斯公式}
	
	{\textbf{全概率公式:} 设试验 $ E $ 的样本空间为 $ S $ , $ A $ 为 $ E $ 的事件, $ B_{1},B_{2}, \cdots ,B_{n} $ 为 $ S $ 的一个划分,且 $ P(B_{i})>0(i=1,2, \cdots ,n) $ ,则: $$ P(A)=P(A|B_{1})P(B_{1})+P(A|B_{2})P(B_{2})+ \cdots +P(A|B_{n})P(B_{n}) $$ }
	
	{\textbf{贝叶斯( $ \texttt{Bayes} $ )公式:} 设试验 $ E $ 的样本空间为 $ S $ , $ A $ 为 $ E $ 的事件, $ B_{1},B_{2}, \cdots ,B_{n} $ 为 $ S $ 的一个划分,且 $ P(A)>0,P(B_{i})>0(i=1,2, \cdots ,n) $ ,则: $$ P(B_{i}|A)= \frac{P(A|B_{i})P(B_{i})}{\sum\limits_{j=1}^{n}P(A|B_{j})P(B_{j})},\ \ i=1,2, \cdots ,n $$ }
	
	\subsection{独立性}
	
	{设 $ A,B $ 两事件如果满足等式 $ P(AB)=P(A)P(B) $ 则称事件 $ A,B $ \textbf{相互独立},简称 $ A,B $ \textbf{独立}.}
	
	\begin{itemize}
		\item [定理:] {
			\begin{enumerate}
				\item [1.] {设 $ A,B $ 是两事件,且 $ P(A)>0 $ .若 $ A,B $ 相互独立,则 $ P(B|A)=P(B) $.反之亦然.}
				\item [2.] {若事件 $ A $ 与 $ B $ 相互独立,则下列各对事件也相互独立: $ A $ 与 $ \bar{B} $ , $ \bar{A} $ 与 $ B $ , $ \bar{A} $ 与 $ \bar{B} $ .}
			\end{enumerate}
		}
	\end{itemize}
	
	{一般,设 $ A_{1},A_{2}, \cdots ,A_{n} $ 是 $ n(n \ge 2) $ 个事件,如果对于其中任意 $ 2 $ 个,任意 $ 3 $ 个, $ \cdots $ ,任意 $ n $ 个事件的积事件的概率,都等于各事件概率之积,则称事件 $ A_{1},A_{2}, \cdots ,A_{n} $ 相互独立.}
	
	\begin{itemize}
		\item [推论:] {
			\begin{enumerate}
				\item [1.] {若事件 $ A_{1},A_{2}, \cdots ,A_{n}(n \ge 2) $ 相互独立,则其中任意 $ k(2 \le k \le n) $ 个事件也是相互独立的.}
				\item [2.] {若 $ n $ 个事件 $ A_{1},A_{2}, \cdots ,A_{n}(n \ge 2) $ 相互独立,则将 $ A_{1},A_{2}, \cdots ,A_{n} $ 中任意多个事件换成它们各自的对立事件,所得的 $ n $ 个事件仍相互独立.}
			\end{enumerate}
		}
	\end{itemize}
	
	\section{第二章\ 随机变量及其分布}
	\subsection{随机变量}
	\subsection{离散型随机变量及其分布率}
	\subsection{随机变量的分布函数}
	\subsection{连续型随机变量及其概率密度}
	\subsection{随机变量的函数的分布}
	\section{第三章\ 多维随机变量及其分布}
	\subsection{二维随机变量}
	\subsection{边缘分布}
	\subsection{条件分布}
	\subsection{相互独立的随机变量}
	\subsection{两个随机变量的函数的分布}
	\section{第四章\ 随机变量的数字特征}
	\subsection{数学期望}
	\subsection{方差}
	\subsection{协方差及相关系数}
	\subsection{矩、协方差矩阵}
	\section{第五章\ 大数定律及中心极限定理}
	\subsection{大数定律}
	\subsection{中心极限定理}
	\section{第六章\ 样本及抽样分布}
	\subsection{随机样本}
	\subsection{直方图和箱线图}
	\subsection{抽样分布}
	\section{第七章\ 参数估计}
	\subsection{点估计}
	\subsection{基于截尾样本的最大似然估计}
	\subsection{估计量的评选标准}
	\subsection{区间估计}
	\subsection{正态总体均值与方差的区间估计}
	\subsection{(0-1)分布参数的区间估计}
	\subsection{单侧置信区间}
\end{document}